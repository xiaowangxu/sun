\documentclass[12pt,oneside,a4paper]{ctexart}
\usepackage{booktabs}
\usepackage{fontspec}
\usepackage{amsmath}
\usepackage{listings}
\usepackage{xcolor}
\usepackage{indentfirst}
\begin{document}

\lstset{numbers=left,
	tabsize=4,
	frame=single,
	xleftmargin=1em,
	basicstyle=\fontspec{Consolas},
	numberstyle=\fontspec{Consolas},
	showstringspaces=false,
	breaklines,
	keywordstyle= \color{ blue!70},
	commentstyle=\color{red!50!green!50!blue!50},
	rulesepcolor= \color{ red!20!green!20!blue!20}
}

\title{计算机图形学 \\ 课程项目}
\author{ 王徐笑风\thanks{学号:18120193 E-mail:2208740924@qq.com}
	\and 凌泽辉\thanks{学号:18120193 E-mail:785896610@qq.com}}
\date{\today}
\maketitle
\newpage
\tableofcontents
\newpage
\begin{abstract}
	查找资料,学习了解三维网格模型的相关知识。完成一个三维网格模型的显示系统。

	数据输入:通过文件读取模型数据

	数据存储:设计程序内用于存储模型数据的数据结构

	数据输出:在窗口界面进行模型显示

	编程实现三维到二维的投影变换计算

	编程实现通过键盘或鼠标驱动模型的平移、缩放及旋转变换

	可以使用开发工具中提供光照函数,若自己编程实现光照计算,则可获得额外加分
\end{abstract}

\section{三维图形的变换与表示}
\subsection{向量与点}
在三维空间中我们常用一个三维向量表示一个点,虽然向量本身只表达长度和方向,他是无关坐标系的,
而点显然是与选取的坐标系是相关的。因此在这里将点理解为在一个给定的坐标系下,原点按某一向量移动后的位置,它写作式(\ref{Point}):
\begin{equation}
	\mathbf{P} = \begin{bmatrix}
		x \\
		y \\
		z
	\end{bmatrix}
	\label{Point}
\end{equation}
\subsection{变换}
对于三维空间中的点,常用到的仿射变换和二维中的类似:平移、旋转和缩放。
\subsubsection{平移}
平移变换是将一个点按一个方向,移动一段距离。考虑到上面我们的点的定义即为在给点的坐标系下,原点按一个向量移动的距离。那么显然平移一个点即将原点平移两次,即点所代表的“向量”和平移向量的共同作用。

对于点$\mathbf{P}$,将其平移$\mathbf{V}$:
\begin{equation*}
	\mathbf{P}=
	\begin{bmatrix}
		\mathbf{P}_x \\
		\mathbf{P}_y \\
		\mathbf{P}_z
	\end{bmatrix}
\end{equation*}
\begin{equation*}
	\mathbf{V}=
	\begin{bmatrix}
		\mathbf{V}_x \\
		\mathbf{V}_y \\
		\mathbf{V}_z
	\end{bmatrix}
\end{equation*}

则有平移后的点$\mathbf{N}$:
\begin{equation*}
	\mathbf{N}=
	\begin{bmatrix}
		\mathbf{P}_x \\
		\mathbf{P}_y \\
		\mathbf{P}_z
	\end{bmatrix}+\begin{bmatrix}
		\mathbf{V}_x \\
		\mathbf{V}_y \\
		\mathbf{V}_z
	\end{bmatrix} =
	\begin{bmatrix}
		\mathbf{P}_x + \mathbf{V}_x \\
		\mathbf{P}_y + \mathbf{V}_y \\
		\mathbf{P}_z + \mathbf{V}_z
	\end{bmatrix}
\end{equation*}
\subsubsection{旋转}
旋转指的是:点以三维空间中的某点为旋转中心,进行旋转,这里简略的认为旋转中心为坐标系原点,即此时的旋转变换是一个特殊的线性变换。
在三维坐标系中对点做按原点的旋转,即是对一个向量进行旋转。只需要求取原基向量 $\hat{i}$、$\hat{j}$、$\hat{k}$ ,在旋转后的 $\hat{i'}$、$\hat{j'}$、$\hat{k'}$,可以得到旋转矩阵:
\begin{equation*}
	\mathbf{RotateMatrix} =
	\begin{bmatrix}
		\hat{i'_x} & \hat{j'_x} & \hat{k'_x} \\
		\hat{i'_y} & \hat{j'_y} & \hat{k'_y} \\
		\hat{i'_z} & \hat{j'_z} & \hat{k'_z}
	\end{bmatrix}
\end{equation*}

则有旋转后的点$\mathbf{N}$:
\begin{equation*}
	\mathbf{N} = \begin{bmatrix}
		\hat{i'_x} & \hat{j'_x} & \hat{k'_x} \\
		\hat{i'_y} & \hat{j'_y} & \hat{k'_y} \\
		\hat{i'_z} & \hat{j'_z} & \hat{k'_z}
	\end{bmatrix} \times \begin{bmatrix}
		\mathbf{P}_x \\
		\mathbf{P}_y \\
		\mathbf{P}_z
	\end{bmatrix}
\end{equation*}

特别的,绕Y轴旋转 $\theta$ 弧度的旋转矩阵,可以这么考虑:首先Y轴显然是不变的,在左手系下从Y轴逆方向向下看,Z轴X轴正好组成一个平面直角坐标系。则 $\hat{i}$ 顺时针旋转过 $\theta$ 弧度后的向量为:
\begin{equation*}
	\hat{i'}=\begin{bmatrix}
		cos(\theta)  \\
		0            \\
		-sin(\theta) \\
	\end{bmatrix}
\end{equation*}

同样的 $\hat{k}$ 旋转过 $\theta$ 弧度后的向量为:
\begin{equation*}
	\hat{k'}=\begin{bmatrix}
		sin(\theta) \\
		0           \\
		cos(\theta) \\
	\end{bmatrix}
\end{equation*}

综上,Y轴旋转矩阵为:
\begin{equation}
	\mathbf{RotateY}(\theta)=\begin{bmatrix}
		cos(\theta)  & 0 & sin(\theta) \\
		0            & 1 & 0           \\
		-sin(\theta) & 0 & cos(\theta)
	\end{bmatrix}
	\label{RotateY}
\end{equation}

类似的我们也可以得到X轴旋转和Z轴旋转矩阵:
\begin{equation}
	\mathbf{RotateX}(\theta)=\begin{bmatrix}
		1 & 0           & 0            \\
		0 & cos(\theta) & -sin(\theta) \\
		0 & sin(\theta) & cos(\theta)
	\end{bmatrix}
	\label{RotateX}
\end{equation}

\begin{equation}
	\mathbf{RotateZ}(\theta)=\begin{bmatrix}
		cos(\theta) & -sin(\theta) & 0 \\
		sin(\theta) & cos(\theta)  & 0 \\
		0           & 0            & 1
	\end{bmatrix}
	\label{RotateZ}
\end{equation}

最后,通过上述(\ref{RotateX})、(\ref{RotateY})、(\ref{RotateZ})的旋转矩阵的复合矩阵即可实现任意的旋转,即:
\begin{equation*}
	\mathbf{Rotate}(\alpha,\beta,\gamma)=\mathbf{RotateZ}(\gamma)\times{}\mathbf{RotateY}(\beta)\times{}\mathbf{RotateX}(\gamma)
\end{equation*}
\subsubsection{缩放}

缩放的实现和旋转矩阵类似,计算出新的 $\hat{i'}$、$\hat{j'}$、$\hat{k'}$ 即可:
\begin{equation*}
	\mathbf{Scale}(\alpha,\beta,\gamma)=\begin{bmatrix}
		\alpha & 0     & 0      \\
		0      & \beta & 0      \\
		0      & 0     & \gamma
	\end{bmatrix}
\end{equation*}

则缩放后的点为:
\begin{equation*}
	\mathbf{N}=\begin{bmatrix}
		\alpha & 0     & 0      \\
		0      & \beta & 0      \\
		0      & 0     & \gamma
	\end{bmatrix}
	\times
	\begin{bmatrix}
		x \\
		y \\
		z
	\end{bmatrix}=
	\begin{bmatrix}
		\alpha{}x \\
		\beta{}y  \\
		\gamma{}z
	\end{bmatrix}
\end{equation*}

\subsubsection{复合变换}
不难看出上述的变换中,旋转和缩放的变换可以简单地实现符合,将旋转矩阵和缩放矩阵进行矩阵乘法复合即可,同时这里两者是可交换的。

但平移变换就无法使用矩阵表达(矩阵变换后的点 $\mathbf{N}$ 的一个分量可以表达为 $\mathbf{N}_x=a\mathbf{P}_x+b\mathbf{P}_y+c\mathbf{P}_z$ 而平移变换可表达为 $\mathbf{N}_x=\mathbf{P}_x+d$ 显然上述公式中 $a=1, \mathbf{P}_y+c\mathbf{P}_z=d$ 显然无法使用一个静态的矩阵实现),因此在描述一个点的平移、旋转和缩放变换时,我们使用下述公式(\ref{Transform_All})
\begin{equation}
	\mathbf{N}=
	(
	\mathbf{Scale}(a,b,c)
	\times
	\mathbf{Rotate}(\alpha,\beta,\gamma)
	)\times
	\mathbf{P}
	+
	\mathbf{V}
	\label{Transform_All}
\end{equation}

但使用上述公式计算是不“简单”的,我们希望能有一种方法通过一次一种计算即可表达上述三种变换,同时又能提高运算的效率。
\subsection{三维的齐次坐标}
在三维中我们无法将平移、旋转和缩放变换由一个矩阵描述。不妨假设我们在四维中,并规定点 $\mathbf{P}$ 在四维中的坐标为:
\begin{equation*}
	\mathbf{P}=
	\begin{bmatrix}
		x \\
		y \\
		z \\
		w=1
	\end{bmatrix}
\end{equation*}

即对于 $\forall{}\mathbf{P}$ 他们位于四维空间中分量 $w=1$ 的一个三维“切片”空间中。而对于 $w\neq 1$ 的点,可以通过计算 $x'=\frac{x}{w}, y'=\frac{y}{w}, z'=\frac{z}{w}$ 来得到对应单位空间中的坐标。

有了三维其次坐标后,我们可以改写公式(\ref{Transform_All})为:
\begin{equation}
	\mathbf{N} =
	\begin{bmatrix}
		a & b & c & d \\
		e & f & g & h \\
		i & j & k & l \\
		0 & 0 & 0 & 1
	\end{bmatrix}_{\mathbf{Transform}}
	\times
	\begin{bmatrix}
		x \\
		y \\
		z \\
		1
	\end{bmatrix}_{\mathbf{P}}
	=
	\begin{bmatrix}
		ax+by+cz+d \\
		ex+fy+gz+h \\
		ix+jy+kz+l \\
		1          \\
	\end{bmatrix}
	\label{Transform_All_New}
\end{equation}

观察公式(\ref{Transform_All_New}),其中 $d, h, l$ 显然对应了平移向量:
\begin{equation*}
	\mathbf{V}=
	\begin{bmatrix}
		d \\
		h \\
		l \\
		1
	\end{bmatrix}
\end{equation*}

它可以写成一个平移矩阵:
\begin{equation}
	\mathbf{Translate}=
	\begin{bmatrix}
		1 & 0 & 0 & d \\
		0 & 1 & 0 & h \\
		0 & 0 & 1 & l \\
		0 & 0 & 0 & 1 \\
	\end{bmatrix}
	\label{Translate_Matrix}
\end{equation}

而将平移矩阵和旋转缩放矩阵复合即可得到公式(\ref{Transform_All_New})中的变换矩阵:
\begin{align}
	\mathbf{Translate}\times \mathbf{Scale}\times \mathbf{Rotation}= &
	\begin{bmatrix}
		1 & 0 & 0 & d \\
		0 & 1 & 0 & h \\
		0 & 0 & 1 & l \\
		0 & 0 & 0 & 1 \\
	\end{bmatrix}_{\mathbf{Translate}}
	\times
	\begin{bmatrix}
		a & b & c & 0 \\
		e & f & g & 0 \\
		i & j & k & 0 \\
		0 & 0 & 0 & 1
	\end{bmatrix}_{\mathbf{Scale}\times \mathbf{Rotation}}         \notag                                       \\
	=                                                                & \begin{bmatrix}
		a & b & c & d \\
		e & f & g & h \\
		i & j & k & l \\
		0 & 0 & 0 & 1
	\end{bmatrix}_{\mathbf{Transform}}\notag
\end{align}

综上我们通过三维空间中的齐次坐标实现了3种变换的复合。当然 $Transform$ 矩阵也能表达切变等仿射变换,由于在三维投影的过程中并不常用,这里就不详细叙述。

\subsubsection{$sunMatLib.h$ 矩阵运算库}
在本项目中,除了窗口创建和像素绘制使用到了第三方软件库,所有的数据结构、矩阵矩阵运算、投影、光栅化、光照等都有我们小组自己实现。这里简单介绍一下整个项目中频繁使用到的矩阵运算类库,见源码:
\begin{lstlisting}[language=c++]
class Vector3
{
	friend class Matrix4;
	
public:
	double _x, _y, _z, _w;
	
public:
	Vector3();
	Vector3(const double &x, const double &y, const double &z);
	Vector3(const Vector3 &copy);
	
	Vector3 &operator=(const Vector3 &copy);
	Vector3 operator+(const Vector3 &b) const;
	Vector3 operator-(const Vector3 &b) const;
	Vector3 operator*(const double &a) const;
	friend Vector3 operator*(const double &a, const Vector3 &b);
	double operator*(const Vector3 &b) const;
	Vector3 &operator*=(const double &a);
	Vector3 &operator+=(const Vector3 &b);
	friend Vector3 operator/(const Vector3 &b, const double &a);
	friend Vector3 &operator/=(Vector3 &b, const double &a);
	double length() const;
	Vector3 normalize() const;
	Vector3 &normalized();
	Vector3 cross(const Vector3 &b) const;
	Vector3 &crossed(const Vector3 &b);
	Vector3 &set(const double &x, const double &y, const double &z);
	
	static Vector3 ONE();
	static Vector3 ZERO();
	static Vector3 UP();
	static Vector3 FRONT();
	static Vector3 RIGHT();
};
\end{lstlisting}
\begin{lstlisting}[language=c++]
class Matrix4
{
	friend class Vector3;

	private:
	double _mat[4][4];

	public:
	Matrix4();
	Matrix4(double copy[4][4]);
	Matrix4(const Matrix4 &copy);

	Matrix4 quickInvert();
	Matrix4 &operator=(const Matrix4 &copy);
	Matrix4 operator*(const Matrix4 &b);
	Vector3 operator*(const Vector3 &b);

	static Matrix4 PROJECTION(const double &aspect_ratio, const double &fov_rad, const double &near_panel, const double &far_panel);
	static Matrix4 ROTATE_X(const double &angle);
	static Matrix4 ROTATE_Y(const double &angle);
	static Matrix4 ROTATE_Z(const double &angle);
	static Matrix4 ROTATE(const double &x, const double &y, const double &z);
	static Matrix4 SCALE(const double &x, const double &y, const double &z);
	static Matrix4 TRANSLATE(const double &x, const double &y, const double &z);
	static Matrix4 POINTAT(Vector3 pos, Vector3 target, Vector3 up);
};
\end{lstlisting}

\subsection{三角面}
虽然三维形体有很多表示方法,但在三维的渲染引擎或是游戏引擎中,最常使用到的还是三角网格模型。

在这种网格模型中一个面一个三维空间中的三角面,它由3个顶点 $\mathbf{P}_1,\mathbf{P}_2, \mathbf{P}_3$ 即可表达。

保留法向信息,三角化的 $.obj$ 文件就是这样一种三角网格模型,在本项目中也该文件格式作为标准。

特别的,在三维渲染中,单个三角面往往还有“朝向”这一属性。背对我们的三角面是不显示的,只有面对我们的才会渲染。这种规定在处理封闭模型和时候会更加有效:首先我们是无法看到封闭模型的内部的,自然无需渲染其三角面的背面;第二,背对我们的三角面不进行渲染也能降低对于性能的需求。因此三角面还存在一个法向量,它垂直于三角面所在的平面,并与三角面朝向同向。

但这不代表我们需要使用除3个三维点以外的数据保存法向信息。在 $.obj$ 中一个三角面 $face$ 由3个顶点 $vertex$ 组成。这三个顶点按原三角面的逆时针排列,其法向满足右手螺旋定则,因此对于1个三角面的有序顶点 $\mathbf{P}_1,\mathbf{P}_2, \mathbf{P}_3$ ,该三角面的单位法向量 $\mathbf{U}$ 为:
\begin{equation*}
	\mathbf{U}=
	\frac{(\mathbf{P}_2 - \mathbf{P}_1)\times
		(\mathbf{P}_3 - \mathbf{P}_1)}{\left\lvert (\mathbf{P}_2 - \mathbf{P}_1)\times
		(\mathbf{P}_3 - \mathbf{P}_1)\right\rvert }
\end{equation*}

在 $sunMatLib.h$ 也有三角面类 $TriFace$ ,它实现了法向的自动计算,在通过顶点构造了三角面后,即可通过 $get_Normal()$ 方法获得单位法向量。具体实现可见源程序代码。
\subsection{网格}
在有了三角面的数据结构后,三角网格模型的最简单的实现就是使用变长数组保存每一个三角面。

在 $sun3D.h$ 中的 $Mesh$ 类就是使用了 $std::vector<TriFace>$ 来保存三角网格模型。
\subsubsection{$.obj$ 文件的加载}

$.obj$ 文件的构成或是说格式,基本上一一对应了上述的三角面、三角网格模型的数据结构。我们先来看一个  $.obj$ 文件实例($untitled1.obj$ 文件有省略):

\begin{lstlisting}
# Blender v2.83.4 OBJ File: ''
# www.blender.org
o Torus
v 1.250000 0.000000 0.000000
v 1.216506 0.125000 0.000000
v 1.125000 0.216506 0.000000
v 1.000000 0.250000 0.000000
......
s off
f 13 2 1
f 2 15 3
f 15 4 3
f 16 5 4
......
\end{lstlisting}

其中“\#”开头的是注释,忽略即可。

“v” 开头的一行定义了一个顶点,后跟三个数值分别表示该顶点 $\mathbf{P}$ 的 $x, y, z$ 分量

“f” 开头的一行定义了一个三角面,后跟三个整数分别表示该三角面的3个顶点 $\mathbf{P}_1, \mathbf{P}_2, \mathbf{P}_3$ 所对应的顶点“v”的序号。例如:上述文件中的第10行 “f 13 2 1”,其中 “2 1” 表明 $\mathbf{P}_2$ 为第5行的顶点, $\mathbf{P}_3$ 为第4行的顶点。

因此,读取 $.obj$ 文件十分简单,设置一个 $std::vector<Vector3>$ 来保存顶点,当读取到一行以“v”开头就构造一个 $Vector3$ 顶点,并放入 $std::vector<Vector3>$ 中,当读取到一行以“f”开头就从 $std::vector<Vector3>$ 中依次取出相应下标的顶点,构造三角面,并放入需要的 $Mesh$ 实例中的 $std::vector<TriFace>$ 即可。
\section{投影}
想要实现三维模型的显示,最重要的是找到一种将三维模型“转换”/“映射”/“投影”到二维平面的方法。
\subsection{正交投影}
\subsection{透视投影}
\section{相机变换}
\subsection{世界坐标系}
\subsection{视口坐标系}
\subsection{坐标系变换}
\section{绘制/光栅化}
\subsection{三角填充}
\subsection{绘制顺序问题}
\subsection{画家算法}
\subsubsection{画家算法的问题}
\subsubsection{改进方向}
\newpage
\section{三维裁剪}
\subsection{性能问题与原因}
\subsection{三角面裁剪}
\subsubsection{近平面裁剪}
\subsubsection{视口裁剪}
\section{基础光照}
\subsection{全局光照}
\subsection{方向光}
\end{document}
