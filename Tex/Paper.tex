\documentclass[12pt,oneside,a4paper]{ctexart}
\usepackage{booktabs}
\usepackage{amsmath}
\usepackage{tikzpfeile}
\begin{document}

\title{计算机图形学}
\author{ 王徐笑风\thanks{学号:18120193 E-mail:2208740924@qq.com}
	\and Ted\thanks{Corresponding author}}
\date{\today}
\maketitle
\newpage
\tableofcontents
\newpage
\begin{abstract}
	在\LaTeX{}中排版中文。
	汉字和English单词混排,通常不需要在中英文之间添加额外的空格。
	当然,为了代码的可读性,加上汉字和 English 之间的空格也无妨。
	汉字换行时不会引入多余的空格。
\end{abstract}

\section{三维图形的表示}
\subsection{向量}
在\TeX{}中排版中文。
汉字和English单词混排,通常不需要在中英文之间添加额外的空格。
当然,为了代码的可读性,加上汉字和 English 之间的空格也无妨。
汉字换行时不会引入多余的空格。

表格

在\LaTeX{}中排版中文。
汉字和English单词混排,通常不需要在中英文之间添加额外的空格。
当然,为了代码的可读性,加上汉字和 English 之间的空格也无妨。
汉字换行时不会引入多余的空格。
\begin{equation}
	\begin{bmatrix}
		1 & 2 \\
		3 & 4
	\end{bmatrix}
	\label{test}
\end{equation}
\begin{equation}
	\mathbf{X}=\int\limits_{a}^{b} \frac{\sqrt[3]{x^2-4ac} }{-2a} dx
\end{equation}
\begin{equation}
	L_{l+1}=AF\left(
	\begin{bmatrix}
			w^l_{1,1}   & w^l_{2,1}   & \cdots & w^l_{a,1}   \\
			w^l_{1,1}   & w^l_{2,2}   & \cdots & w^l_{a,2}   \\
			\vdots      & \vdots      & \ddots & \vdots      \\
			w^l_{1,b-1} & w^l_{2,b-1} & \cdots & w^l_{a,b-1} \\
			w^l_{1,b}   & w^l_{2,b}   & \cdots & w^l_{a,b}   \\
		\end{bmatrix}
	\cdot
	\begin{bmatrix}
			H^l_1     \\
			H^l_2     \\
			\vdots    \\
			H^l_{a-1} \\
			H^l_{a}   \\
		\end{bmatrix}
	+
	\begin{bmatrix}
			\theta{}^{l+1}_1     \\
			\theta{}^{l+1}_2     \\
			\vdots               \\
			\theta{}^{l+1}_{b-1} \\
			\theta{}^{l+1}_{b}   \\
		\end{bmatrix}
	\right)\label{test2}
\end{equation}
公式(\ref{test}),是一个矩阵
\section{投影}
\section{绘制}
\subsection{三角填充}
\subsection{绘制顺序问题}
\subsection{画家算法}
\subsubsection{画家算法——类比}
\subsubsection{画家算法的问题}
\subsubsection{改进方向}
\section{相机变换}
\section{三维裁剪}
\subsection{性能问题与原因}
\subsection{平面表达}
\subsection{三角面与平面裁剪}
\subsection{视口裁剪}
\section{基础光照}
\subsection{全局光照}
\subsection{方向光}
\subsection{冯氏光照模型}
\end{document}
