\documentclass[12pt,oneside,a4paper]{ctexart}
\usepackage{booktabs}
\usepackage{amsmath}
\usepackage{indentfirst}
\begin{document}

\title{计算机图形学 \\ 课程项目}
\author{ 王徐笑风\thanks{学号:18120193 E-mail:2208740924@qq.com}
	\and 凌泽辉\thanks{学号:18120193 E-mail:785896610@qq.com}}
\date{\today}
\maketitle
\newpage
\tableofcontents
\newpage
\begin{abstract}
	查找资料,学习了解三维网格模型的相关知识。完成一个三维网格模型的显示系统。

	数据输入:通过文件读取模型数据

	数据存储:设计程序内用于存储模型数据的数据结构

	数据输出:在窗口界面进行模型显示

	编程实现三维到二维的投影变换计算

	编程实现通过键盘或鼠标驱动模型的平移、缩放及旋转变换

	可以使用开发工具中提供光照函数,若自己编程实现光照计算,则可获得额外加分
\end{abstract}

\section{三维图形的表示与变换}
\subsection{向量与点}
在三维空间中我们常用一个三维向量表示一个点,虽然向量本身只表达长度和方向,他是无关坐标系的,
而点显然是与选取的坐标系是相关的。因此在这里将点理解为在一个给定的坐标系下,原点按某一向量移动后的位置,它写作式(\ref{Point}):
\begin{equation}
	\mathbf{P} = \begin{bmatrix}
		x \\
		y \\
		z
	\end{bmatrix}
	\label{Point}
\end{equation}
\subsection{变换}
对于三维空间中的点,常用到的仿射变换和二维中的类似:平移、旋转和缩放。
\subsubsection{平移}
平移变换是将一个点按一个方向,移动一段距离。考虑到上面我们的点的定义即为在给点的坐标系下,原点按一个向量移动的距离。那么显然平移一个点即将原点平移两次,即点所代表的“向量”和平移向量的共同作用。

对于点$\mathbf{P}$,将其平移$\mathbf{V}$:
\begin{equation*}
	\mathbf{P}=
	\begin{bmatrix}
		\mathbf{P}_x \\
		\mathbf{P}_y \\
		\mathbf{P}_z
	\end{bmatrix}
\end{equation*}
\begin{equation*}
	\mathbf{V}=
	\begin{bmatrix}
		\mathbf{V}_x \\
		\mathbf{V}_y \\
		\mathbf{V}_z
	\end{bmatrix}
\end{equation*}

则有平移后的点$\mathbf{N}$:
\begin{equation*}
	\mathbf{N}=
	\begin{bmatrix}
		\mathbf{P}_x \\
		\mathbf{P}_y \\
		\mathbf{P}_z
	\end{bmatrix}+\begin{bmatrix}
		\mathbf{V}_x \\
		\mathbf{V}_y \\
		\mathbf{V}_z
	\end{bmatrix} =
	\begin{bmatrix}
		\mathbf{P}_x + \mathbf{V}_x \\
		\mathbf{P}_y + \mathbf{V}_y \\
		\mathbf{P}_z + \mathbf{V}_z
	\end{bmatrix}
\end{equation*}
\subsubsection{旋转}
旋转指的是:点以三维空间中的某点为旋转中心,进行旋转,这里简略的认为旋转中心为坐标系原点,即此时的旋转变换是一个特殊的线性变换。
在三维坐标系中对点做按原点的旋转,即是对一个向量进行旋转。只需要求取原基向量 $\hat{i}$、$\hat{j}$、$\hat{k}$ ,在旋转后的 $\hat{i'}$、$\hat{j'}$、$\hat{k'}$,可以得到旋转矩阵:
\begin{equation*}
	RotateMatrix =
	\begin{bmatrix}
		\hat{i'_x} & \hat{i'_y} & \hat{i'_z} \\
		\hat{j'_x} & \hat{j'_y} & \hat{j'_z} \\
		\hat{k'_x} & \hat{k'_y} & \hat{k'_z}
	\end{bmatrix}
\end{equation*}

则有旋转后的点$\mathbf{N}$:
\begin{equation*}
	\mathbf{N} = \begin{bmatrix}
		\hat{i'_x} & \hat{i'_y} & \hat{i'_z} \\
		\hat{j'_x} & \hat{j'_y} & \hat{j'_z} \\
		\hat{k'_x} & \hat{k'_y} & \hat{k'_z}
	\end{bmatrix} \times \begin{bmatrix}
		\mathbf{P}_x \\
		\mathbf{P}_y \\
		\mathbf{P}_z
	\end{bmatrix}
\end{equation*}

特别的,绕Y轴旋转 $\theta$ 弧度的旋转矩阵,可以这么考虑:首先Y轴显然是不变的,在左手系下从Y轴逆方向向下看,Z轴X轴正好组成一个平面直角坐标系。则 $\hat{i}$ 顺时针旋转过 $\theta$ 弧度后的向量为:
\begin{equation*}
	\hat{i'}=\begin{bmatrix}
		cos(\theta)  \\
		0            \\
		-sin(\theta) \\
	\end{bmatrix}
\end{equation*}

同样的 $\hat{k}$ 旋转过 $\theta$ 弧度后的向量为:
\begin{equation*}
	\hat{k'}=\begin{bmatrix}
		sin(\theta) \\
		0           \\
		cos(\theta) \\
	\end{bmatrix}
\end{equation*}

综上,Y轴旋转矩阵为:
\begin{equation}
	RotateY(\theta)=\begin{bmatrix}
		cos(\theta)  & 0 & sin(\theta) \\
		0            & 1 & 0           \\
		-sin(\theta) & 0 & cos(\theta)
	\end{bmatrix}
	\label{RotateY}
\end{equation}

类似的我们也可以得到X轴旋转和Z轴旋转矩阵:
\begin{equation}
	RotateX(\theta)=\begin{bmatrix}
		1 & 0           & 0            \\
		0 & cos(\theta) & -sin(\theta) \\
		0 & sin(\theta) & cos(\theta)
	\end{bmatrix}
	\label{RotateX}
\end{equation}

\begin{equation}
	RotateZ(\theta)=\begin{bmatrix}
		cos(\theta) & -sin(\theta) & 0 \\
		sin(\theta) & cos(\theta)  & 0 \\
		0           & 0            & 1
	\end{bmatrix}
	\label{RotateZ}
\end{equation}

最后,通过上述(\ref{RotateX})、(\ref{RotateY})、(\ref{RotateZ})的旋转矩阵的复合矩阵即可实现任意的旋转,即:
\begin{equation*}
	Rotate(\alpha,\beta,\gamma)=RotateZ(\gamma)\times{}RotateY(\beta)\times{}RotateX(\gamma)
\end{equation*}
\subsubsection{缩放}

缩放的实现和旋转矩阵类似,计算出新的 $\hat{i'}$、$\hat{j'}$、$\hat{k'}$ 即可:
\begin{equation*}
	Scale(\alpha,\beta,\gamma)=\begin{bmatrix}
		\alpha & 0     & 0      \\
		0      & \beta & 0      \\
		0      & 0     & \gamma
	\end{bmatrix}
\end{equation*}

则缩放后的点为:
\begin{equation*}
	\mathbf{N}=\begin{bmatrix}
		\alpha & 0     & 0      \\
		0      & \beta & 0      \\
		0      & 0     & \gamma
	\end{bmatrix}
	\times
	\begin{bmatrix}
		x \\
		y \\
		z
	\end{bmatrix}=
	\begin{bmatrix}
		\alpha{}x \\
		\beta{}y  \\
		\gamma{}z
	\end{bmatrix}
\end{equation*}

\subsubsection{复合变换}
不难看出上述的变换中,旋转和缩放的变换可以简单地实现符合,将旋转矩阵和缩放矩阵进行矩阵乘法复合即可,同时这里两者是可交换的。

但平移变换就无法使用矩阵表达(矩阵变换后的点 $\mathbf{N}$ 的一个分量可以表达为 $\mathbf{N}_x=a\mathbf{P}_x+b\mathbf{P}_y+c\mathbf{P}_z$ 而平移变换可表达为 $\mathbf{N}_x=\mathbf{P}_x+d$ 显然上述公式中 $a=1, \mathbf{P}_y+c\mathbf{P}_z=d$ 显然无法使用一个静态的矩阵实现),因此在描述一个点的平移、旋转和缩放变换时,我们使用下述公式(\ref{Transform_All})
\begin{equation}
	\mathbf{N}=
	(
	Scale(a,b,c)
	\times
	Rotate(\alpha,\beta,\gamma)
	)
	+
	\mathbf{V}
	\label{Transform_All}
\end{equation}

但使用上述公式计算是不“简单”的,我们希望能有一种方法通过一次一种计算即可表达上述三种变换,同时又能提高运算的效率。
\subsection{三维的齐次坐标}
在三维中我们无法将平移、旋转和缩放变换由一个矩阵描述。不妨假设我们在四维中,并规定点 $\mathbf{P}$ 在四维中的坐标为:
\begin{equation*}
	\mathbf{P}=
	\begin{bmatrix}
		x \\
		y \\
		z \\
		1
	\end{bmatrix}
\end{equation*}

即对于 $\forall{}\mathbf{P}$ 他们位于四维空间中分量 $z=1$ 的一个三维“切片”空间中。
\subsection{三角面}
\subsection{网格}
\subsubsection{.obj文件的加载}
\section{投影}
\subsection{正交投影}
\subsection{透视投影}
\section{相机变换}
\subsection{世界坐标系}
\subsection{视口坐标系}
\subsection{坐标系变换}
\section{绘制/光栅化}
\subsection{三角填充}
\subsection{绘制顺序问题}
\subsection{画家算法}
\subsubsection{画家算法的问题}
\subsubsection{改进方向}
\section{三维裁剪}
\subsection{性能问题与原因}
\subsection{三角面裁剪}
\subsubsection{近平面裁剪}
\subsubsection{视口裁剪}
\section{基础光照}
\subsection{全局光照}
\subsection{方向光}
\end{document}
